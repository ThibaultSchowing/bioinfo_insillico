\documentclass[10pt,a4paper]{report}
\usepackage[utf8]{inputenc}

% \usepackage{ngerman}  % german documents
\usepackage{graphicx}  % import graphics einbinden
\usepackage{listings}  % support source code listing
\usepackage{amsmath}  % math stuff
\usepackage{amssymb} % 
\usepackage{a4wide} % wide pages
\usepackage{fancyhdr} % nice headers
\usepackage{float}
\usepackage{longtable}
\usepackage{xcolor}

\definecolor{darkpastelgreen}{rgb}{0.01, 0.75, 0.24}
\definecolor{spirodiscoball}{rgb}{0.06, 0.75, 0.99}
\definecolor{smalt}{rgb}{0.0, 0.2, 0.6}
\definecolor{armygreen}{rgb}{0.29, 0.33, 0.13}
\definecolor{awesome}{rgb}{1.0, 0.13, 0.32}
\definecolor{bittersweet}{rgb}{1.0, 0.44, 0.37}
\definecolor{bananayellow}{rgb}{1.0, 0.88, 0.21}
\definecolor{blue}{rgb}{0.0, 0.0, 1.0}
\definecolor{red}{rgb}{1.0, 0.0, 0.0}
\definecolor{green}{rgb}{0.0, 1.0, 0.0}



\lstset{basicstyle=\footnotesize,language=Python,breaklines=true,numbers=left, numberstyle=\tiny, stepnumber=5,firstnumber=0, numbersep=5pt} % set up listings
\pagestyle{fancy}             % header
\setlength{\parindent}{0pt}   % no indentation

\usepackage[pdfpagemode=None, colorlinks=true,  % url coloring
           linkcolor=blue, urlcolor=blue, citecolor=blue, plainpages=false, 
           pdfpagelabels,unicode]{hyperref}
           
% change enums style: first level (a), (b), (c)           
\renewcommand{\labelenumi}{(\alph{enumi})}
\renewcommand{\labelenumii}{(\arabic{enumii})}

%lecture name
\newcommand{\lecture}{
	Bioinformatics Practicals In Sillico
}           

%assignment iteration
\newcommand{\assignment}{
	Project
}


%set up names, matricle number, and email
\newcommand{\authors}{
  \begin{tabular}{rl}
    \href{mailto:thibault.schowing@unifr.ch}{Thibault Schowing}\\
    \href{mailto:lio_roh@students.unibe.ch}{Lionel Rohner}\\
    \href{mailto:alain.rohrbasser.unifr.ch}{Alain Rohrbasser}\\
    \href{mailto:rares.cristea@unifr.ch}{Rares Cristea}
  \end{tabular}
}


\begin{document}
\title{\Large \lecture \\ \textbf{\normalsize \assignment}}
\author{\authors}

\setlength \headheight{25pt}
\fancyhead[R]{\begin{tabular}{r}\lecture \\ \assignment \end{tabular}}
\fancyhead[L]{\authors}


\setcounter{section}{3} % modify for later sheets, i.e. 2, 3, ...
%\section{Introduction to Python and some Network Properties} % optional, note that section invocation sets the section counter + 1, so adapt the setcounter command
\maketitle

%WIEBKE !!! READ THIS !!! 
% Usefull with Latex and maths: a list of the symbols ! https://reu.dimacs.rutgers.edu/Symbols.pdf


\chapter*{Yeast Genome Analysis}



File types used
\begin{itemize}
	\item \textbf{FA} The files with the .fa extension store FASTA format sequences. In this project the .fa file contains the reference genome.  
	\item \textbf{GTF} The Gene transfer format (GTF) is a file format used to hold information about gene structure. It is a tab-delimited text format based on the general feature format (GFF), but contains some additional conventions specific to gene information. A significant feature of the GTF that can be validated: given a sequence and a GTF file, one can check that the format is correct. This significantly reduces problems with the interchange of data between groups.
	\item \textbf{VCF} The Variant Call Format stores the gene sequence variation. By using the variant call format only the variations need to be stored along with a reference genome which make the file less redundant.
	\item \textbf{BAM} Binary Alignment Map (BAM) is the comprehensive raw data of genome sequencing; it consists of the lossless, compressed binary representation of the Sequence Alignment Map. BAM is the compressed binary representation of SAM (Sequence Alignment Map). BAM is in compressed BGZF format.
\end{itemize}


\chapter*{Arabidopsis Thaliana Genome Analysis}
Part 2 


\chapter*{Lactobacillus Heleveticae Genome Analysis}
Part 3 



\end{document}