\documentclass[10pt,a4paper]{article}
\usepackage[utf8]{inputenc}

% \usepackage{ngerman}  % german documents
\usepackage{graphicx}  % import graphics einbinden
\usepackage{listings}  % support source code listing
\usepackage{amsmath}  % math stuff
\usepackage{amssymb} % 
\usepackage{a4wide} % wide pages
\usepackage{fancyhdr} % nice headers
\usepackage{float}
\usepackage{longtable}
\usepackage{xcolor}
\usepackage{cite}
\usepackage{fancyhdr}
\usepackage{tabularx}


\definecolor{darkpastelgreen}{rgb}{0.01, 0.75, 0.24}
\definecolor{spirodiscoball}{rgb}{0.06, 0.75, 0.99}
\definecolor{smalt}{rgb}{0.0, 0.2, 0.6}
\definecolor{armygreen}{rgb}{0.29, 0.33, 0.13}
\definecolor{awesome}{rgb}{1.0, 0.13, 0.32}
\definecolor{bittersweet}{rgb}{1.0, 0.44, 0.37}
\definecolor{bananayellow}{rgb}{1.0, 0.88, 0.21}
\definecolor{blue}{rgb}{0.0, 0.0, 1.0}
\definecolor{red}{rgb}{1.0, 0.0, 0.0}
\definecolor{green}{rgb}{0.0, 1.0, 0.0}



\lstset{basicstyle=\footnotesize,language=Python,breaklines=true,numbers=left, numberstyle=\tiny, stepnumber=5,firstnumber=0, numbersep=5pt} % set up listings
\pagestyle{fancy}             % header

\setlength{\parindent}{3pt}   % no indentation

\usepackage[pdfpagemode=None, colorlinks=true,  % url coloring
           linkcolor=blue, urlcolor=blue, citecolor=blue, plainpages=false, 
           pdfpagelabels,unicode]{hyperref}
           
% change enums style: first level (a), (b), (c)           
\renewcommand{\labelenumi}{(\alph{enumi})}
\renewcommand{\labelenumii}{(\arabic{enumii})}

%lecture name
\newcommand{\lecture}{
	Bioinformatics Practicals In Sillico
}           

%assignment iteration
\newcommand{\assignment}{
	BC-7107
}


%set up names, matricle number, and email
\newcommand{\authors}{
  \begin{tabular}{rl}
    \href{mailto:thibault.schowing@unifr.ch}{Thibault Schowing}\\
    \href{mailto:lio_roh@students.unibe.ch}{Lionel Rohner}\\
    \href{mailto:alain.rohrbasser.unifr.ch}{Alain Rohrbasser}\\
    \href{mailto:rares.cristea@unifr.ch}{Rares Cristea}
  \end{tabular}
}

\setlength \headheight{25pt}
\fancyhead[R]{\begin{tabular}{r}\lecture \\ \assignment \end{tabular}}
\fancyhead[L]{HS-2019}

\title{\Large Bioinformatics Practicals In Sillico \\ \textbf{\normalsize BC-7107}}
\author{\authors}



\begin{document}

\maketitle
\newpage

\part*{Introduction}



%File types used
%\begin{itemize}
%	\item \textbf{FA} The files with the .fa extension store FASTA format sequences. In this project the .fa file contains the reference genome.  
%	\item \textbf{GTF} The Gene transfer format (GTF) is a file format used to hold information about gene structure. It is a tab-delimited text format based on the general feature format (GFF), but contains some additional conventions specific to gene information. A significant feature of the GTF that can be validated: given a sequence and a GTF file, one can check that the format is correct. This significantly reduces problems with the interchange of data between groups.
%	\item \textbf{VCF} The Variant Call Format stores the gene sequence variation. By using the variant call format only the variations need to be stored along with a reference genome which make the file less redundant.
%	\item \textbf{BAM} Binary Alignment Map (BAM) is the comprehensive raw data of genome sequencing; it consists of the lossless, compressed binary representation of the Sequence Alignment Map. BAM is the compressed binary representation of SAM (Sequence Alignment Map). BAM is in compressed BGZF format.
%\end{itemize}


\newpage
\part*{Yeast Genome Analysis}

\section*{Introduction}

\paragraph{Biological introduction}The budding Yeast Saccharomyces cerevisiae is a common organism used for genetics manipulation. This organism is well conserved among the eukaryote and can be used correlate with human pathways. With a genome with 16 chromosomes (haploid, Mat a or $ \alpha $) or 32 chromosomes (diploid). 99\% of the genome is without introns, make this organism handy to manipulate. 12 million bases pair and contains between 5 800 to 6 572 genes [TODO REF]. The homology with human is estimate to 23\%, which is a good candidate for preliminary studies regarding human pathways. The short mating time and growth is also short. Thus, the identification of potential mutant is grandly enhanced. This is a single eukaryotic organism with a division cycle of 90 minutes. Through the process of budding in which smaller daughter cells pinch, or bud, off the mother cell. Due to the microscopic size ($~$5 microM, between bacteria and human cell size) and simple growth environment, yeasts are inexpensive and easy to grow in silico. Saccharomyces cerevisiae is also no-pathogen, and forms colonies on agar plates in the laboratory in a few days with no special incubators required (best grow at 30 $ \deg $). 

\paragraph{\textit{tom1}} 

\section*{Methods}


\newpage
\part*{Arabidopsis Thaliana Genome Analysis}

\section*{Introduction}
\paragraph{Biological introduction} 


\section*{Methods}


\newpage
\part*{Lactobacillus Heleveticus Genome Assembly}
\section*{Introduction}

%TODO table abreviation (LAB)(PGHs)



The diverse bacteria involved in cheese production are essential for the texture and taste development but also, during the ripening process, the microbial changes helps to kill pathogens and reduce spoilage micro-organisms. \textit{Lactobacillus helveticus} is a thermophilic lactic acid bacterium (LAB) used in the dairy industry as a starter or an adjunct culture for cheese manufacture\cite{jebava_nine_2011}. By releasing \textbf{peptidoglycan hydrolases}(PGHs), it has the ability to digest the bacterial cell wall (gram+) inducing death of surrounding bacteria but also its autolysis. \\

The genomic plasticity of \textit{Lactobacillus helveticus} leads to a high variation in PGHs activity from one strain to another. In a previous study, the activity of a PGH with an estimated size of 30kDa was tested by zymography in nine strains of \textit{Lactobacillus helveticus} of which six were sequenced (see figure \ref{fig:zymography}). Two phenotypes were shown: phenotype A exhibits PGH activity (strains \textbf{FAM8102c1c1}, \textbf{FAM23285} and \textbf{FAM19191}) and phenotype B does not (strains \textbf{FAM22016}, \textbf{FAM1450} and \textbf{FAM1213}).\\

The aim of this work was to detect potential genomic differences involved in the two different phenotypes by sequencing, assembling and compare the genome of the six strains using a previously annotated reference genome of \textit{Lactobacillus helveticus} (\href{https://www.ncbi.nlm.nih.gov/genome/?term=NC_010080}{NC\_010080}). A potential candidate present only in the strains expressing a PGHs activity suggests that it might have been acquired by a viral insertion. 

  
%TODOOOOOOOOOOOOOOO

% BLASTP the protein sequence to see it comes from a phage !

\section*{Methods}

%TODO cite SOAPdenovo, Spades and Abyss
\paragraph{Sequencing and genome assembly}
The six \textit{Lactobacillus helveticus} strains \textbf{FAM8102c1c1}, \textbf{FAM23285}, \textbf{FAM19191}, \textbf{FAM22076}, \textbf{FAM1450}, \textbf{FAM1213}  were sequenced by Illumina sequencing. The following tasks were performed using the cluster provided by the University of Bern.  \textit{FastQC}\cite{noauthor_babraham_nodate} was used to check the quality of the reads and \textit{Trimmomatic}\cite{bolger_trimmomatic:_2014} to filter out bad quality reads. \textit{SOAPdenovo}\cite{noauthor_soapdenovo:_nodate} as well as \textit{Spades}\cite{noauthor_spades_nodate} were used to perform the genome assembly with the reads of each strains. For \textit{SOAPdenovo} the k-mer sizes were set to 95, 85, 75 and 65. For \textit{Spades} k-mere sizes were set to 21, 33, 55, 77 and 99 (default values). The four assemblies of SOAPdenovo and the assembly of Spades were compared using Abyss with a maximum number of contigs set to 1000. The best genome assemblies with the bigger N50 and a approximate genome size of 20Mbp (Genome size of \textit{Lactobacillus helveticus}) were then chosen.


\paragraph{Genome annotation and pan-genome analysis}
We used the \textit{PROKKA} pipeline\cite{seemann_prokka:_2014} to annotate the genome of the six best assemblies and the reference genome for \textit{Lactobacillus helveticus} \href{https://www.ncbi.nlm.nih.gov/genome/?term=NC_010080}{NC\_010080}. \textit{PROKKA} is an automated pipeline that annotates prokaryotic genomes. It locates open reading frames ans RNA regions on contigs and translates it to protein sequences, searching for protein homologues in public databases. The resulting standards \href{https://www.ensembl.org/info/website/upload/gff.html}{.gff} files containing the annotated genome for each strain are then used by \textit{Roary}\cite{page_roary:_2015} to generate a pan-genome of the six strains. The result was then visualized with \textit{Phandango}\cite{hadfield_phandango:_2018} allowing visualisation of phylogenetic tree, associated metadata and genomic information.


\paragraph{Extraction of the genes for each phenotypes} Grep\cite{noauthor_grep1_nodate} was applied to the files generated by \textit{Roary} to extract the nine PHG's \cite{jebava_nine_2011} labelled "Lhv\_" with \textit{PROKKA}. The set of genes found in strains expressing phenotype A was then compared to the set of gene showing phenotype B. In table \ref{tab:resultPGHexpr} we have the two PGHs present only in the three strains expressing the PGHs activity. The nucleotide sequences were then converted to amino acid sequences for further comparison. 
%TODOAAAAAAAAAAAAAAAAAAAAa

%TODO INSERT FIGURE
\begin{figure}
	\centering
	\includegraphics[width=0.6\linewidth]{img/zymography}
	\caption[The two phenotypes expressed by the six strains]{Phenotype A is expressing an active peptidoglycan hydrolase and phenotype B is not.}
	\label{fig:zymography}
\end{figure}

\section*{Results}



%TODO add equivalent in kDa somewhere

%TODO we have actually 2 candidates -> 1899 is pesent in more than just the 3 here
\begin{table}[htbp]
	\centering
	\begin{tabularx}{\linewidth}{|X|X|X|X|X|X|}
		\hline
		\textbf{Gene} & \textbf{Annotation} & \textbf{Avg group size nuc} & \textbf{FAM19191\_ 1K} & \textbf{FAM23285\_ 1K} & \textbf{FAM8102\_ 1K}\\
		 \hline
		%group\_1899 & Lhv\_2053 Lysin (L.crispatus) pseudogene in L.helveticus & 830/ 30 kDa & FAM19191\_ 1K\_00615 & FAM23285\_ 1K\_00607 & FAM8102\_ 1K\_00746 \\
		%\hline
		group\_2348 & Lhv\_2053 Lysin (L.crispatus) pseudogene in L.helveticus & 1121/ 41 kDa & FAM19191\_ 1K\_00069 & FAM23285\_ 1K\_00060 & FAM8102\_ 1K\_00069 \\
		\hline
		group\_2372 & Lhv\_2053 Lysin (L.crispatus) pseudogene in L.helveticus & 893/ 33 kDa & FAM19191\_ 1K\_00397 & FAM23285\_ 1K\_00499 & FAM8102\_ 1K\_00565 \\
		\hline	
	\end{tabularx}
	\caption{Genes present only in the three strains with a PGH activity.}
	\label{tab:resultPGHexpr}
\end{table}

According to figure \ref{fig:zymography}, the protein involved is approximately 30kDa thus matches with group 2372. Using BLASTp \cite{altschul_gapped_1997}, the protein was searched to be a particular lysin (\href{https://www.ncbi.nlm.nih.gov/protein/1325986555}{WP\_101853908.1}) encoded by the pneumococcal bacteriophage Cp-1\cite{martin_pneumococcal_1998}. 












\newpage
\bibliography{mybib}{}
\bibliographystyle{ieeetr}

\end{document}